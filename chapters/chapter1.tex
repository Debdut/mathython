
\chapter{Creating Mathython Evnironment}

\section{Introduction}

In this chapter we will create the {\color{cyan} Mathython}  Environment. I have been using GNU/Linux environment while writing this book
and will assume that you are also running some GNU/Linux distribution or atleast some Unix-type operating system. I hope that
does not deter you from reading this book! {\color{cyan} Don't worry! this chapter includes instructions for setting up a GNU/Linux distribution.}
Or if you can import my instructions for use in windows, I won't mind. If you are in MacOS, feel happy because you are already
in a Unix-type environment but still there will be some little differences.

Choose any GNU/Linux distribution based on Debian/Ubuntu, some of the preferred distributions for beginners are :

\begin{itemize}
\item \href{http://trisquel.info/}{GNU Trisquel}
\item \href{http://elementary.io}{Elementary OS (Mac-Like)}
\item \href{http://xubuntu.org}{Xubuntu}
\item \href{http://www.ubuntu.com}{Ubuntu}
\item \href{http://linuxmint.com}{Linux Mint (Windows-Like)}
\end{itemize}

Out of this, I prefer {\color{cyan} Elementary OS}\footnote{I have some special instructions for Elementary OS \href{http://itsfoss.com/top-ten-installing-elementary-os-luna/}{here} to make it the nicest workplace. Experiment with it and have some fun!} for beginners. Note The instructions in the whole book for all of this distributions is same.

\section{Starting Python}

Open Terminal or console, type {\color{magenta} python} followed by {\color{magenta} <Enter>} and you will probably see this:

\inputpythoncodefile{src_org/chapter1/pyintro.py}

Did you see the same thing (without colors)? Yeppy! You are running python.
If not, then probably your distribution does not have python installed. Go to \href{https://www.python.org/ftp/python/2.7.9/Python-2.7.9.tar.xz}{Downloads} and save the file. Change to dirctory of download and do the following to install python :

\begin{bashcommands}
$tar -xvzf Python-2.7.9.tar.xz
$cd Python-2.7.9
$./configure
$make
$./python
$sudo make install
\end{bashcommands}

Ok, now you must be able to open python by typing {\color{magenta} python} in terminal.

\section{Installing Python Packages}