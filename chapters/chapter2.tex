
\chapter{Python \& Numbers}

\section{Hello Numbers}

Lets get our hands dirty! We start with this {\color{magenta} Hello Numbers} example :

\inputpythoncodefile{src_org/chapter2/hellonum.py}

Write or copy this code into file {\color{magenta} hellonum.py,}  save it and run it. You will probably get this :

\inputtextfile{src_org/chapter2/hellonum.txt}

Ok! lets understand what that program does. The command {\color{magenta} print} takes numbers as arguments
 and prints them to the file {\color{magenta} stdout}. This file is the Standard Output file, which in turn is printed
on the screen before us. This print command can also print strings as shown in the following example :

\inputpythoncodefile{src_org/chapter2/favnum.py} 

Guess what this outputs,

\inputtextfile{src_org/chapter2/favnum.txt}

\subsection{Numbers and Strings}
\subsubsection{Numbers}
Python generally separates numbers in two ways :

\begin{itemize}
\item Integers $(\cdots -2, -1, 0, 1, 2, 3, \cdots)$ or plain integers as type {\color{cyan} int}
\item Floating point numbers, eg. $(1.0, 1.2, 1.3, 0.0, 1.3332)$ , as type {\color{cyan} float} to represent decimals upto a certain precision.
\end{itemize}

Python also has other types such as {\color{cyan} Decimal}, {\color{cyan} Fraction}, {\color{cyan} complex}
to handle other types of numbers. We will go through those later in this book.

\subsubsection{Strings and Characters}

On the other hand, python needs to handle characters and other things. Single characters, eg. (a, b, c, A, Z) are handled
by {\color{cyan} char} type in python.

But to handle sequence to characters, python has String or {\color{cyan} str} type, eg. (``Hello Numbers", `My favorite number is 7.0'). Strings are enclosed in either
double quotes ``'' or single quotes `'. There are some special character such as :

\begin{itemize}
\item ` ' or space character
\item \textbackslash n or newline character
\item \textbackslash t or tab character
\item \textbackslash b or backspace character
\item \textbackslash v or vertical tab character
\end{itemize}

A program to understand special characters :
After you copy this program, please make sure that you write the print command line is a single line, otherwise, you get a error. Here it is shown in many lines for convenience.
Special strings kown as {\color{cyan} DocStrings} with three quotes, ``` ''', support this feature.

\inputpythoncodefile{src_org/chapter2/special.py}

The ` ' or space character insert single blank in between other characters, while \textbackslash t or tab character insert a tab, and with \textbackslash n newline character
we go to new line. The backspace character does not show the previous character.

\inputtextfile{src_org/chapter2/special.txt}

