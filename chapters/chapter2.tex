
\chapter{Python \& Numbers}

\section{Basics}

Lets get our hands dirty! We start with this {\color{magenta} Hello Numbers} example :

\inputpythoncodefile{src_org/chapter2/hellonum.py}

Write or copy this code into file {\color{magenta} hellonum.py,}  save it and run it. You will probably get this :

\inputtextfile{src_org/chapter2/hellonum.txt}

Ok! lets understand what that program does. The command {\color{magenta} print} takes numbers as arguments
 and prints them to the file {\color{magenta} stdout}. This file is the Standard Output file, which in turn is printed
on the screen before us. This print command can also print strings as shown in the following example :

\inputpythoncodefile{src_org/chapter2/favnum.py} 

Guess what this outputs,

\inputtextfile{src_org/chapter2/favnum.txt}

\subsection{Numbers and Strings}
\subsubsection{Numbers}
Python generally separates numbers in two ways :

\begin{itemize}
\item Integers $(\cdots -2, -1, 0, 1, 2, 3, \cdots)$ or plain integers as type {\color{cyan} int}
\item Floating point numbers, eg. $(1.0, 1.2, 1.3, 0.0, 1.3332)$ , as type {\color{cyan} float} to represent decimals upto a certain precision.
\end{itemize}

{\color{cyan} int} have the capability to handle numbers from $-2^{31}$ to $2^{31}-1$ . For integers larger than that are handled by {\color{cyan} long} type. For example, (100000000000000L, 1L). The letter {\color{cyan} L} at the end denotes type. Operations involving {\color{cyan} long} type numbers may output in {\color{cyan} long} type even if they are in range of {\color{cyan} int} type.

Python also has other types such as {\color{cyan} Decimal}, {\color{cyan} Fraction}, {\color{cyan} complex}
to handle other types of numbers. We will go through those later in this book.

\subsubsection{Strings and Characters}

On the other hand, python needs to handle characters and other things. Single characters, eg. (a, b, c, A, Z) are handled
by {\color{cyan} char} type in python.

But to handle sequence to characters, python has String or {\color{cyan} str} type, eg. (``Hello Numbers", `My favorite number is 7.0'). Strings are enclosed in either
double quotes ``'' or single quotes `'. There are some special character such as :

\begin{itemize}
\item ` ' or space character
\item \textbackslash n or newline character
\item \textbackslash t or tab character
\item \textbackslash b or backspace character
\item \textbackslash v or vertical tab character
\end{itemize}

A program to understand special characters :
After you copy this program, please make sure that you write the print command line is a single line, otherwise, you get a error. Here it is shown in many lines for convenience.
Special strings known as {\color{cyan} Verbatim Strings} with three quotes, ``` ''', support this feature.  These are strings in which escape sequences (such as \textbackslash n) won't be interpreted, line breaks will be preserved and quotation marks don't have to be masked. They are enclosed with triple double quotation marks. Possible indentations after a line break are part of the string as well.

\inputpythoncodefile{src_org/chapter2/special.py}

The ` ' or space character insert single blank in between other characters, while \textbackslash t or tab character insert a tab, and with \textbackslash n newline character
we go to new line. The backspace character does not show the previous character.

\inputtextfile{src_org/chapter2/special.txt}

\subsection{Booleans}

You may or may not know that there is a lot math about booleans. In python, we call them {\color{cyan} bool} type. It have two values {\color{cyan} true} and {\color{cyan} false} . For mathematical purposes, we use 1 for {\color{cyan} true} and 0 for {\color{cyan}}.
We will talk about them later.

\section{Python as calculator}

Open python console and follow me :

\inputpythoncodefile{src_org/chapter2/democal.py}

After we write each line, we press {\color{magenta} <Enter>} , get the result and continue.
Some operations in python :

\begin{itemize}
\item + or addition operator; it is a binary operator, i.e, takes two inputs and gives single output.
\item * or multiplication operator; it is a binary operator, i.e, takes two inputs and gives single output.
\item - or subtraction operator; it is a binary operator, i.e, takes two inputs and gives single output.
\item / or division operator; it is a binary operator, i.e, takes two inputs and gives single output.
\item ** or exponentiation operator; it is a binary operator, i.e, takes two inputs and gives single output. (Simply, power operator, for eg. $2^2$ or 2 to the power 2)
\item \% or modulus operator; it is a binary operator, i.e, takes two inputs and gives single output. It is the remainder operator.
\end{itemize}

Note : All this operators other than last one work with {\color{cyan} int}, {\color{cyan} long}, {\color{cyan} Fraction}, {\color{cyan} float} and {\color{cyan} complex}. The modulus character works with only {\color{cyan} int} and {\color{cyan} long} type. The + operation is also used for strings to conacatenate or merge. Other operators in the list do not work with strings. We will talk about this topic later.

\subsection{Addition}

\inputpythoncodefile{src_org/chapter2/addition.py}

Lets proceed to subtraction.

\subsection{Subtraction}

\inputpythoncodefile{src_org/chapter2/subtraction.py}

\subsection{Multiplication}

\inputpythoncodefile{src_org/chapter2/multiply.py}


\subsection{Division}

\inputpythoncodefile{src_org/chapter2/dividion.py}

In the second example, we see that 10/4 = 2. Actually it gives the quotient of 10 divided by 4.
But 10.0/4 = 2.5 since 10.0 is a {\color{cyan} float} type, python performs a decimal type division.
In the last example we see that we get a error message on dividing by 0 as it is not possible.


